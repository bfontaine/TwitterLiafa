\documentclass[]{article}

\usepackage[francais]{babel}

\usepackage{amssymb,amsmath}
\usepackage{ifxetex,ifluatex}
\usepackage{graphics,graphicx}
\usepackage{pstricks,pst-node,pst-tree}
\ifxetex
  \usepackage{fontspec,xltxtra,xunicode}
  \defaultfontfeatures{Mapping=tex-text,Scale=MatchLowercase}
\else
  \ifluatex
    \usepackage{fontspec}
    \defaultfontfeatures{Mapping=tex-text,Scale=MatchLowercase}
  \else
    \usepackage[utf8]{inputenc}
  \fi
\fi
%\ifxetex
%  \usepackage[setpagesize=false, % page size defined by xetex
%              unicode=false, % unicode breaks when used with xetex
%              xetex,
%              colorlinks=true,
%              linkcolor=blue]{hyperref}
%\else
  \usepackage[unicode=true,
              colorlinks=true,
              linkcolor=blue]{hyperref}
%\fi
\hypersetup{breaklinks=true, pdfborder={0 0 0}}
\setlength{\parindent}{0pt}
\setlength{\parskip}{6pt plus 2pt minus 1pt}
\setlength{\emergencystretch}{3em}  % prevent overfull lines
\setcounter{secnumdepth}{0}

\title{Rapport de stage}
\author{Baptiste Fontaine}
\date{21 mai 2012}

\begin{document}
\maketitle
\tableofcontents
\newpage
\section{Introduction}

Ce stage m'a été proposé par Christophe Prieur alors que je ne trouvais
pas de matière pour valider mon UE libre\footnote{Unité d'Enseignement
 hors cursus au choix, à suivre pendant un semestre}. Il s'est déroulé
pendant la quasi-totalité du semestre, durée pendant laquelle j'ai
principalement travaillé chez moi, avec un rendez-vous hebdomadaire avec
C. Prieur pour avoir un suivi régulier. J'ai aussi pu travailler
directement au LIAFA plusieurs fois pendant les vacances de Pâques.

C. Prieur et Stéphane Raux étudient la façon dont interagissent les
utilisateurs de \href{https://twitter.com/}{Twitter}, et cherchent à 
expliquer les motivations des utilisateurs qui ``retweetent''
(\emph{i.e.} qui re-postent du contenu publié par d'autres utilisateurs,
et donc qui contribuent à sa dissémination sur le réseau). Pour ce
faire, ils ont extrait des ensembles d'utilisateurs du réseau en
fonction d'une URL que chacun de ces utilisateurs avait ``tweeté''.
Chacun de ces ensembles est décrit par plusieurs graphes --- $G_{0}$, $G_{f}$,
$G_{RT}$ --- dont les définitions sont données plus loin.

\subsection{Définitions}

\begin{description}
\item[Twitter :]
Réseau social créé en 2006 sur lequel les utilisateurs peuvent publier
de courts messages publics, et s'abonner aux messages d'autres
utilisateurs. Ces messages sont appelés des ``\emph{Tweets}'', et
``\emph{suivre}'' quelqu'un signifie être abonné à ses \emph{Tweets}.
Lorsqu'un utilisateur poste un message à propos d'une information ou qui
contient une URL, on dit qu'il \emph{tweet} cette information ou URL. Le
site propose une API pour les développeurs, permettant de récupérer
toutes les informations sur les \emph{Tweets} et les utilisateurs
facilement. La version Web de Twitter est d'ailleurs basée sur sa propre
API.

\item[Retweet :]
\emph{Tweet} d'un utilisateur qui est re-publié par un autre
utilisateur, afin que ses abonnés voient le \emph{Tweet} originel. Le
verbe associé est ``\emph{retweeter}''. Lorsqu'on \emph{retweet} le
\emph{Retweet} d'un utilisateur, notre message est affiché comme étant
le \emph{Retweet} du \emph{Tweet} originel. Ainsi, si l'utilisateur B
retweet A, puis C retweet B, le \emph{Tweet} de C est affiché comme
étant un \emph{Retweet} de A, sans mention de l'utilisateur B.

\item[$G_{0}$ :]
Graphe dont les noeuds sont les utilisateurs qui ont tweeté une URL
donnée, et les liens les interactions\footnote{Une interaction est
 matérialisée soit par un \emph{Retweet}, soit par la mention du nom
 d'un autre utilisateur dans un \emph{Tweet}} qu'ils ont eu entre eux
avant le début de la diffusion de l'URL. Dans les figures, un lien de
$G_{0}$ sera représenté en rouge.

\item[$G_{f}$ :]
Graphe des abonnés (``\emph{followers}''), \emph{i.e.} dont chaque lien
matérialise une relation d'abonnement sur le réseau; Un lien de A vers B
indique que A suit B. Il est censé représenter les liens d'abonnements
entre les utilisateurs avant le début de la diffusion de l'URL\footnote{Les
 données relatives aux abonnements ont en fait été collectée après.}.
Dans les figures, un lien de $G_{f}$ est représenté en bleu.

\item[$G_{RT}$ :]
Graphe des \emph{Retweets}, \emph{i.e.} dont chaque lien matérialise un
\emph{Retweet}; un lien de A vers B indique que A a reweeté B. Chacun de
ces \emph{Retweets} contient l'URL étudiée. Dans les figures, un lien de
$G_{RT}$ est représenté en noir.

\end{description}
\section{Objectifs}

Lorsque j'ai commencé le stage, S. Raux avait déjà extrait les
informations pour 124 ensembles d'utilisateurs. Pour chacun de ces
ensembles était disponibles un fichier de graphe $G_{0}$ et deux fichiers
au format JSON, l'un étant simplement le résultat de la recherche
effectuée sur une URL via l'API, l'autre une liste datée des
\emph{Retweets} entre ces utilisateurs, permettant de construire $G_{RT}$.

Mes objectifs étaient les suivants :

\begin{itemize}
\item
 Dans une première partie, récupérer les informations concernant les
 abonnements des utilisateurs, afin de générer les $G_{f}$.
\item
 Dans une seconde partie, extraire des informations chiffrées sur les
 graphes générés, afin de permettre une étude statistique de ceux-ci.
\end{itemize}
\section{Méthodologie}

Durant tout le stage, j'ai utilisé des scripts en \textbf{Ruby} pour
automatiser les tâches répétitives et/ou trop longues à faire à la main.
J'ai aussi utilisé le logiciel \href{https://gephi.org}{\textbf{Gephi}}
pour visualiser les graphes, et \textbf{R} pour les statistiques.

\subsection{Technique}

Afin de représenter des graphes simplement, j'ai utilisé des listes de
\emph{hashs}\footnote{équivalent en Ruby des dictionnaires de Python ou
 des \emph{HashMap}s de Java} pour les noeuds et les liens de chaque
graphe. J'ai utilisé le format
\href{http://guess.wikispot.org/The\_GUESS\_.gdf\_format}{\emph{GDF}}
pour stocker les graphes, car c'est un format qui a l'avantage d'être
peu verbeux (deux lignes plus une ligne par noeud ou lien) et facile à 
générer et à parser. Pour l'occasion, j'ai mis en ligne une petite
bibliothèque sous forme de gem\footnote{paquet logiciel utilisé pour
 partager des modules en Ruby, équivalent des \emph{eggs} de Python},
\href{https://rubygems.org/gems/graphs}{\texttt{graphs}}, permettant de
manipuler des graphes et de les stocker au format \emph{GDF}\footnote{La
\emph{gem} a été téléchargée plus de 400 fois depuis sa mise en ligne}.

\subsection{Première partie}

Le principal obstacle pour la première partie était la limite horaire du
nombre de requêtes à l'API de Twitter\footnote{les requêtes sont
 limitées à 350 par heure pour une application authentifiée}. En effet,
l'ensemble des graphes représente 7000 utilisateurs uniques; récupérer
les identifiants de leurs abonnés demande une requête par tranche de
50000\footnote{seuls une vingtaine de comptes dépassaient ce seuil}.
J'ai écris un script pour récupérer ces informations, et S. Raux m'a
proposé de le faire tourner en continu sur un serveur chez
\href{http://fr.linkfluence.net/}{Linkfluence}, ce qui a permis de
récupérer les informations sous 48h. Le script générait un fichier par
utilisateur listant ses abonnés. J'ai ensuite croisé ces listes avec les
informations des graphes pré-existants pour générer les $G_{f}$.

\subsection{Seconde partie}

Pour la seconde partie, j'ai dû rajouter des fonctions à ma petite
bibliothèque afin de pouvoir faire de l'arithmétique sur les graphes :
unions, intersections, \emph{XOR}. Ainsi, générer l'intersection de deux
graphes devenait aussi simple que la ligne de code suivante :
\begin{verbatim}
GDF::load("graph1.gdf") & GDF::load("graph2.gdf")
\end{verbatim}
J'ai dû aussi gérer certains cas où les attributs des noeuds entre
différents graphes n'étaient pas les mêmes, il fallait dans ce cas
n'effectuer des comparaisons que sur les attributs communs des noeuds.

En utilisant ces fonctions, j'ai pû générer des pourcentages de
recouvrement entre certaines parties de graphes (par exemple la
proportion de $G_{RT}$ qui est dans $G_{0}$, ou la proportion de $G_{RT}$ qui est
dans $G_{f}$ mais pas dans $G_{0}$, etc).

Pour terminer, il a fallu calculer les taux d'explications possibles
pour les \emph{Retweets}. Par exemple, un \emph{Retweet} peut avoir eu
lieu parce que la personne qui a retweeté a déjà interagi avec la
source du \emph{Tweet}, dans ce cas, un lien de A vers B dans $G_{RT}$
existe aussi dans $G_{0}$. Cela peut aussi être parce que A a déjà 
interagi avec quelqu'un qui a retweeté B auparavant, ou parce que A
suit B tout simplement, ou encore parce que A suit quelqu'un qui a retweeté
B auparavant.

\section{Résultats}

L'ensemble des chiffres donnés ici concernent un ensemble de 124
graphes\footnote{Chaque graphe étant un ensemble d'utilisateurs qui ont
 tweeté une URL donnée. Les graphes existent en trois variantes: $G_{0}$,
 $G_{f}$ et $G_{RT}$.}, 7215 utilisateurs uniques, et 9947 \emph{Tweets}
(dont 4446 \emph{Retweets}).

\subsection{Recouvrements}

Pour chaque graphe, quatre intersections ont été calculées, avec à chaque fois
le taux de recouvrement par rapport à $G_{RT}$.

\begin{figure}[!h]
\begin{center}
\begin{tabular}{|l|r|r|r|r|}
\hline
Intersection&\multicolumn{4}{c|}{Taux de recouvrement par rapport à $G_{RT}$}\\
\hline
&premier quartile&médiane&moyenne&troisième quartile\\
\hline
$G_0$ et $G_{RT}$ & 60.28\% & 72.22\% & 70.85\% & 82.85\%\\
\hline
$G_f$ et $G_{RT}$ & 11.50\% & 41.50\% & 42.29\% & 72.06\%\\
\hline
$G_0$ et $G_f$ et $G_{RT}$ & 7.55\% & 34.74\% & 34.28\% & 56.73\%\\
\hline
$G_f$ et $G_{RT}$ sans $G_0$ & 0\% & 4.37\% & 8.00\% & 12.33\%\\
\hline
\end{tabular}
\end{center}
\caption{Taux de recouvrement des graphs}
\end{figure}

On remarque ainsi que $70.85\%$\footnote{En moyenne} des \emph{Retweets} sont
faits par des gens qui avaient déjà interagi avec l'auteur du \emph{Tweet},
mais que seulement $41.5\%$\footnote{idem} sont issus de personnes abonnées
à celui-ci, ce qui montre bien qu'il y a eu une dissémination du \emph{Tweet}
originel sur le réseau, en dehors de ses propres abonnés. Le faible taux de
recouvrement entre $G_f$ et $G_{RT}$ sans $G_0$ ($8\%$ en moyenne) confirme que
très peu de \emph{Retweets} sont fait par des gens qui suivent mais n'avaient
pas interagi avec l'auteur du \emph{Tweet}.

\subsection{Différentes explications possibles}

Quatre explications différentes pour un \emph{Retweet} étaient
envisagées :

\begin{itemize}
\item
 l'utilisateur qui retweet a déjà interagi avec l'auteur du
 \emph{Tweet} originel, autrement dit il existe un lien direct dans
 $G_{0}$.
\begin{figure}[!h]
\begin{center}
$
\psmatrix[colsep=1.5cm,rowsep=1.5cm,mnode=circle]
A&B
\ncline{->}{1,1}{1,2}
\ncarc[arcangle=20,linecolor=red,linewidth=2pt]{->}{1,1}{1,2}
\endpsmatrix
$
\end{center}
\caption{Explication 1}
\end{figure}
\item
 l'utilisateur qui retweet a déjà interagi avec quelqu'un qui a déjà 
 retweeté le même \emph{Tweet}, mais n'a jamais interagi avec l'auteur originel.
\begin{figure}[!h]
\begin{center}
$
\psmatrix[colsep=1.5cm,rowsep=1.5cm,mnode=circle]
&C\\
A&&B
\ncline{->}{2,1}{2,3}
\ncarc[arcangle=20,linewidth=2pt]{->}{1,2}{2,3}
\ncarc[arcangle=20,linecolor=red,linewidth=2pt]{->}{2,1}{1,2}
\endpsmatrix
$
\end{center}
\caption{Explication 2}
\end{figure}
\item
 l'utilisateur qui retweet suis l'auteur du \emph{Tweet} originel,
 autrement dit il existe un lien direct dans $G_{f}$.
\begin{figure}[!h]
\begin{center}
$
\psmatrix[colsep=1.5cm,rowsep=1.5cm,mnode=circle]
A&B
\ncline{->}{1,1}{1,2}
\ncarc[arcangle=20,linecolor=blue,linewidth=2pt]{->}{1,1}{1,2}
\endpsmatrix
$
\end{center}
\caption{Explication 3}
\end{figure}
\item
 l'utilisateur qui retweet suis quelqu'un qui a déjà retweeté le même
 \emph{Tweet}, mais ne suit pas l'auteur originel.
\begin{figure}[!h]
\begin{center}
$
\psmatrix[colsep=1.5cm,rowsep=1.5cm,mnode=circle]
&C\\
A&&B
\ncline{->}{2,1}{2,3}
\ncarc[arcangle=20,linewidth=2pt]{->}{1,2}{2,3}
\ncarc[arcangle=20,linecolor=blue,linewidth=2pt]{->}{2,1}{1,2}
\endpsmatrix
$
\end{center}
\caption{Explication 4}
\end{figure}
\end{itemize}
Les différents scénarios explicatifs ont été étudiés indépendamment les
uns des autres.

\subsubsection{Premier scénario}

Dans le premier scénario, un utilisateur retweet quelqu'un avec qui il a
déjà interagi\footnote{Ce qui correspond à un recouvrement entre $G_0$
et $G_{RT}$}. C'est celui le plus souvent rencontré; en moyenne,
70.8\% des \emph{Retweets} sont issus de personnes qui avaient déjà 
interagi avec l'auteur du \emph{Tweet}. La médiane est à 72.2\%, le
premier quartile à 60.2\% et le troisième à 82.8\%.

\subsubsection{Deuxième scénario}

Dans le deuxième scénario, un utilisateur A a déjà interagi avec un
autre, B, qui a retweeté l'auteur du tweet originel, C; A n'a pas eu
d'interaction directe avec C. On explique donc son \emph{Retweet} par le
fait qu'il a vu le \emph{Retweet} de B avant. En moyenne, 24.9\% des
\emph{Retweets} sont expliqués par ce scénario. La médiane est à 17.1\%,
les premier et troisième quartiles à 5.5\% et 41.6\% respectivement.

\subsubsection{Troisième scénario}

Le troisième scénario est similaire au premier, sauf qu'ici, on
n'explique pas le \emph{Retweet} par une interaction mais par un
abonnement : A retweet B parce qu'il est abonné à ses \emph{Tweets}. En
moyenne, 42.3\% des \emph{Retweets} sont expliqués par ce scénario. La
médiane est à 41.5\%, les premier et troisième quartiles à 11.5\% et
72.0\% respectivement.

\subsubsection{Quatrième scénario}

Le quatrième scénario est similaire au premier, sauf que, comme dans le
troisième, on n'explique pas ici le \emph{Retweet} par une interaction
mais par un abonnement\footnote{Ce qui correspond à un recouvrement entre $G_f$
et $G_{RT}$} : A est abonné à B mais pas à C, B retweet C
\emph{puis} A retweet C \emph{via} B. A a donc vu le \emph{Tweet}
originel de C retweeté par B auquel il est abonné, avant de lui-même le
retweeter. Ce scénario explique 13.2\% des \emph{Retweets} en moyenne.
La médiane est à 7.1\%, les premier et troisième quartiles à 0\% et
17.6\% respectivement.

\subsubsection{Résumé}

\begin{figure}[h]
\begin{center}
\begin{tabular}{|c|r|r|r|r|}
\hline
Scénario&premier quartile&médiane&moyenne&troisième quartile\\
\hline
1&60.28\%&72.22\%&70.85\%&82.85\%\\
\hline
2&5.56\%&17.16\%&24.96\%&41.66\%\\
\hline
3&11.50\%&41.50\%&42.29\%&72.06\%\\
\hline
4&0.00\%&7.14\%&13.23\%&17.68\%\\
\hline
\end{tabular}
\end{center}
\caption{Résumé des explications}
\end{figure}
\section{Conclusion}

Ce stage m'a permis de mettre en pratique les connaissances apprises en cours
ou par moi-même\footnote{Le langage Ruby, par exemple}, de façon autonome,
et avec la présentation de résultats. Il m'a permis de travailler avec des
graphes, ce qui est (entre autres) l'objet du cours d'algorithmique en L3,
que je suivrais l'année prochaine. J'ai aussi pu avoir un aperçu de ce que
peut être la recherche en informatique.

\end{document}